\subsection{Classification}

\subsubsection{Discrete Fourier Transform (DFT))}

% Si DFT toma, en su naive version \Thetta{N^2}, Cooley-Tukey realizaron FFT con una complejidad de \Thetta(N*logN).

% TO-DO: Change sub sub section name (?)

Fourier Transforms are usually defined in the continuous realm of numbers, but for effects of computation, we will deal with it is reciprocal: the Discrete Fourier transform.

\subsubsection{Algorithms to solve DFT}

\textbf{Fast Fourier Transform (FFT)}   
\noindent
Cooley and Tukey, in 1964, developed an algorithm to calculate the finite Fourier transform in a fast way, with a complexity of $O(n*logn)$, opposite to the naive version that has a complexity of $O(n^2)$. In each case, $n$ describes data points, where is recommend to use $n = 2^m$ points. The Cooley-Tukey algorithm is known as Fast Fourier transform. \\

\begin{comment}
\noindent
Next we present the pseudocode for FFT on listing \ref{related-works-listing-fft-pseudocode} and the implementation on listing \ref{related-works-listing-fft-implementation}:

\lstinputlisting[language={}, caption="FFT pseudocode", style=listingcode, label=related-works-listing-fft-pseudocode]{sections/2related_works/source_code/fft.txt}

\lstinputlisting[language=C++, caption="FFT implementation", style=listingcode, label=related-works-listing-fft-implementation]{sections/2related_works/source_code/fft_recursive.cpp}

\bigskip \noindent
\textbf{Sparse Fast Fourier Transform}
\noindent
\end{comment}
